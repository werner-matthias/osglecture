\def\packagename{langselect}
% \iffalse meta-comment
%
%<*pkg>
%%  (C) 2022-2025 Matthias Werner
%%
%%  Package langselect - generate documents in different languages
%%                       from a common source.
%%
%</pkg>
% \fi% meta
%<*driver>
\def\nameofplainTeX{plain}
\ifx\fmtname\nameofplainTeX
\input docstrip.tex
\keepsilent
\askforoverwritefalse
\preamble
----------------------------------------------------------------
langselect -- multiple language versions from a common source
E-mail: matthias.werner@informatik.tu-chemnitz.de
Released under the LaTeX Project Public License v1.3c or later
See http://www.latex-project.org/lppl.txt
----------------------------------------------------------------

\endpreamble
\postamble
This work is "maintained" (as per LPPL maintenance status) by
Matthias Werner.

This work consists of the file  langselect.dtx
and the derived files           langselect.pdf and
                                langselect.sty.
\endpostamble

\generate{
  \file{\packagename.sty}{\from{\packagename.dtx}{pkg}}
}
  \expandafter\endbatchfile
\fi
% Patch cnltx-doc bundle
\typeout{*** PATCH}
\AddToHook{class/scrartcl/after}{
  \NewCommandCopy{\originalAfterPackage}{\AfterPackage}
  \RenewDocumentCommand{\AfterPackage}{t! m m}{
    \IfBooleanTF{#1}{#3}{%
      \originalAfterPackage{#2}{#3}
    }
  }
}
\makeatletter
\AddToHook{package/cnltx-example/after}{%
\renewcommand*\cnltx@exe{\ShellEscape}%
}

%\def\cnltxmdframed{\cnltx@mdframed@options}
\makeatother
\ExplSyntaxOn
\sys_if_engine_luatex:TF{
  \def\error{}
  \RequirePackage{luacode}
}{
  
  \def\error{\errmessage{LuaLaTeX~is~required~to~build~this~documentation.}\endinput}
}
\error
\ExplSyntaxOff
\RequirePackage{luatex85}
\RequirePackage{shellesc}
\RequirePackage{multicol}
\documentclass[
load-preamble-,
]{cnltx-doc}
\usepackage[oldstyle]{libertine}
\setmonofont[
Scale = MatchLowercase ,
Ligatures = {NoCommon,NoRequired,NoContextual}
]{DejaVu Sans Mono}
\usepackage{microtype}
\usepackage{enumitem}
\usepackage{readprov}
\usepackage{hyperref}
\usepackage{multicol}
\usepackage{luacode}
\usepackage{listings}
\lstset{language=TeX}
\mdfdefinestyle{cnltxmdframed}{
  backgroundcolor = cnltxbg ,
 linecolor = cnltx ,
 roundcorner = 5pt
}
% --- Mini-Autodoc: comments = doku --------
\begin{luacode*}
-- autodoc: % => text; %<*pkg>..%</pkg> => Code
--   - Kommentarzeilen (%...) als Fließtext (kein verbatim),
--   - Nicht-%-Zeilen als verbatim-Code gesetzt.
-- Außerhalb von Guards: %... als Fließtext, sonst ignorieren.

local function strip_pct(s) return (s:gsub("^%%+%s?","")) end

function autodoc_process(fname)
  tex.print("\\typeout{*********** Input: '"..fname.."'}")
  local f = io.open(fname,"r")
  if not f then tex.error("AutoDocInput: cannot open "..fname); return end
  local in_driver = false
  local in_guard  = false  -- z.B. *package*
  local in_vb     = false  -- verbatim läuft?

  local function vb_open()
    if not in_vb then
      tex.print("\\color{blue}\\begin{verbatim}")
      in_vb=true
      print("START VERB")
    end
  end
  local function vb_close()
    if in_vb then
      tex.print("\\end{verbatim}\\color{black}")
      in_vb=false
      print("END VERB")
    end
  end

  for line in f:lines() do
    -- Driver auslassen
    if line:match("^%%<%*driver>") then
      in_driver=true
      print("START DRIVER")
      goto continue
    end
    if line:match("^%%</driver>")   then
    in_driver=false
      print("END DRIVER")
      goto continue
    end
    if in_driver then goto continue end

    -- Beliebige Guard-Blöcke wie %<*package>
    local guard_open = line:match("^%%<%*(%w+)>")
    if guard_open then
       in_guard=true;
       print("START PACKAGE")
       vb_close();
       goto continue
    end
    if line:match("^%%</%w+>") then
      in_guard=false
      vb_close()
      print("END PACKAGE")
      goto continue
    end
    if in_guard then
      if line:match("^%%") then
        -- Fließtext innerhalb des Guards
        vb_close()
        tex.print(strip_pct(line))
        print("*T* ".. strip_pct(line))
      else
        -- echter Code innerhalb des Guards
        vb_open()
        tex.print(line)
        print("*C* ".. line)
      end
    else
      -- Außerhalb der Guards: nur Doku drucken
      if line:match("^%%") then
        vb_close()
        tex.print(strip_pct(line))
        print("*** ".. strip_pct(line))
      else
        -- nackte Zeilen außerhalb ignorieren
      end
    end
    ::continue::
  end

  vb_close()
  f:close()
end
\end{luacode*}

\newcommand{\AutoDocInput}[1]{\directlua{autodoc_process("#1")}}
\usepackage[
languages={de,en},
targetlang={job=2}
]{\packagename}
% --------------------------------------------------------------------

\ReadFileInfos{\packagename.sty}
\def\thepkg{\pkg*{\packagename}}
\setcnltx{
  name  = \packagename,
  package  = \packagename,
  version  = \UseVersionOf{\packagename.sty},
  date     = \UseDateOf{\packagename.sty},
  title    = The \pkg*{langselect} Package,
  info     = \ldeen{Generierung verschiedener Sprachversionen aus einer Quelle}{Generate different language versions from a source},
  authors  = {Matthias Werner[matthias.werner@informatik.tu-chemnitz.de]},
  abstract={
    \ldeen{
      Dieses Paket unterstützt die Generierung von Dokumenten in zwei oder drei Sprachvarianten aus einem gemeinsamen
      Quelldokument.  
      Das Paket gehört zum \pkg*{osglecture}-Bundle und wird in die \cls*{osglecture}-Klasse
      integriert und vom Buildscript \code{ollm} unterstützt, kann aber auch eigenständig
      genutzt werden.
    }{
      This package supports the generation of documents in two or three language versions from a common
      source document. 
      The package is part of the \pkg{osglecture} bundle and is integrated into the \cls*{osglecture} class
      and supported by the build script \code{ollm}, but can also be used independently.
    }
  },
  url      =https://github.com/tuc-osg/osglecture,
  build-title
}
\def\thepkg{\pkg*{\packagename}}
 
\begin{document}

\section{\ldeen{Einleitung}{Introduction}}
\label{sec:intro}
\ldeen{Mitunter werden mehrere Sprachversionen des gleichen Dokuments benötigt.
Im Anwendungsgebiet des \pkg*{osglecture}-Bündel sind dies beispielsweise die
Lehrskripte, Vorlesungsfolien und Handouts einer Lehrveranstaltung.
Es fällt häufig schwer, mehrere parallel existierende Sprachversionen konsistent zu halten.
Diese Aufgabe ist etwas einfacher, wenn alle Versionen in einem gemeinsamen
Quelldokument enthalten sind.

Dieses Paket unterstützt die Generierung solcher Sprachversionen aus einem
gemeinsamen \LaTeX-Quelldokument indem es Makros für Sprachvarianten und
Mechanismen zur Auswahl der Zielsprache bereitstellt.
}{
Sometimes, multiple language versions of the same document are required.
In the application area of the \pkg*{osglecture} bundle, these are, for example, the
lecture notes, lecture slides, and handouts for a course.
It is often difficult to keep multiple parallel language versions consistent.
This task is somewhat easier if all versions are contained in a common
source document.

This package supports the generation of such language versions from a
common \LaTeX\ source document by providing macros for language variants and
mechanisms for selecting the target language.
}

\subsection{\ldeen{Alternativen}{Alternatives}}
\label{sec:alternatives}
\ldeen{Es gibt eine Reihe von Alternativen zu \thepkg.}{The are a number of
  alternatives to \thepkg.}
\begin{description}
  \item [\ldeen{Adhoc-Macros}{Ad hoc macros}]\ldeen{Man kann relativ einfach adhoc Lösungen schaffen, wie z.\,B.}{It
      is relatively easy to create ad hoc solutions, such as the following}
    \begin{example}[code-only]
        \newif\ifenglish
        \englishtrue
        \ifenglish
            Welcome!
        \else
            ¡Bienvenido!
        \fi
    \end{example}
    \ldeen{
      Tatsächlich folgt \thepkg\ grundsätzlich diesem Ansatz, automatisiert aber die
      Erstellung der Sprachmakros, vereinfacht die gemeinsame Nutzung
      von nichtsprachabhängigen Elementen wie z.\,B. Formeln und ermöglicht die
      Auswahl der Sprachvariante über verschiedene Methoden.
    }{
      In fact, \thepkg\ basically follows this approach, but automates the
      creation of language macros, simplifies the sharing
      of non-language-dependent elements such as formulas, and enables the
      selection of the target language by different approaches.
    }
  \item[\pkg{multilang}]
    \ldeen{
      Das ausgeklügelte Paket \needpackage{multilang} von \name{Richard Grewe} erstellt
      Sprachversionen von gegebenen Makros, wobei die aktive Sprache direkt aus
      \pkg*{Babel} oder \pkg*{Polyglossia} übernommen wird.
      \thepkg\ hat ein ähnliches Ziel, aber ein anderes (aus Sicht des
      Paketautoren: praktischeres) Interface.
      Falls Sie aber den Ansatz von \thepkg\ nicht mögen, sollten Sie sich
      unbedingt \pkg*{multilang} anschauen.
    }{
      The sophisticated package \needpackage{multilang} by \name{Richard Grewe} creates
      language versions of given macros, whereby the active language is taken directly from
      \pkg*{Babel} or \pkg*{Polyglossia}.
      \thepkg\ has a similar goal, but a different (from the package author's
      point of view: more practical) interface. 
      However, if you don't like the approach of \thepkg\, you should definitely
      take a look at \pkg*{multilang}. 
    }
  \item[\pkg{comment}]
    \ldeen{
      \name{Victor Eijkhout} hat des Paket \needpackage{comment} geschrieben,
      das auf einfache Weise ermöglicht, nur bestimmte Abschnitte im Dokument
      auszugeben. Dies kann auch sehr gut für eine Sprachauswahl genutzt werden.
    }{
      \name{Victor Eijkhout} wrote the package \needpackage{comment},
      which makes it easy to output only certain sections of the document.
      This can also be used very well for language selection.
    }
  \item[\pkg{translations}/\pkg{translator}/\pkg{xt\_capts}]
    \ldeen{
      Diese drei Pakete
      \needpackage{translations}/\needpackage{translator}/\needpackage[/macros/latex/contrib/xtcapts]{xt\_capts}
      addressieren Programmierer von \LaTeX-Paketen, um fixe Textelemente wie
      die Überschriften von Verzeichnissen zu internationalisieren. Für Autoren mehrsprachiger Dokumente sind diese Pakete
      nur bedingt geeignet.
    }{
      These three packages are intended for programmers of \LaTeX\ packages to internationalize fixed
      text elements such as the headings of tables of contents.
      These packages are only of limited use for authors of multilingual documents.
  }
\end{description}
\subsection{\ldeen{Sprachen}{Languages}}
\label{sec:languages}
Das Konzept von \thepkg\ beinhaltet die Nutzung von Sprachen auf verschiedenen
Ebenen mit unterschiedlicher Wirkung. Um Verwechselungen zu vermeiden, wollen
wir hier die Begriffe eindeutig definieren.
\begin{enumerate}
  \item \ldeen*{
      Die Menge \emph{möglicher} Zielsprachen des Dokuments. Ein Dokument, das
      \thepkg\ nutzt, enthält typischerweise Abschnitte in verschiedenen
      Zielsprachen. Die möglichen Zielsprachen werden über eine Paketoption
      gesetzt, siehe Abbschnitt~@1. Im weiteren Text dieser Dokumentation
      nennen wir die möglichen Zielsprachen \emph{Auswahlsprachen}, da für die
      \LaTeX-Übersetzung eine dieser Sprachen ausgewählt wird.
    }{
      The set of \emph{possible} target languages for the document. A document that
      uses \thepkg\ typically contains sections in different target
      languages. The possible target languages are set via a package option, see
      Section~@1. In the rest of this documentation,
      we refer to the possible target languages as \emph{selectable languages},
      since one of these languages is selected for the \LaTeX\ translation.
    }{\ref{sec:options}}
  \item
    \ldeen{%
      Die tatsächliche bei einer \LaTeX-Übersetzung gebrauchte Zielsprache.
      Diese muss eine der Sprachen aus 1. sein. Typischerweise wird der
      Übersetzungsvorgang für jede der tatsächlichen Zielsprachen einzeln
      ausgeführt.
      Für die Auswahl der tatsächlichen Zielsprache gibt es mehrere Methoden,
      siehe Abschnitt}{%
      The actual target language used in a LaTeX translation.
      This must be one of the languages from 1. Typically, the  translation
      process is performed separately for each of the actual target languages. 
      There are several methods for selecting the actual target language;  see
      section}~\ref{sec:targetlang}.
  \item
    \ldeen{
      Die Sprache, in der Textteile oder einzelne Wörter tatsächlich geschrieben
      sind. Dies ist das Konzept von Sprache, das Pakete wie Babel oder
      Polyglossia haben.\footnote{Wobei auch in diesen Pakten "`Sprache"' auch
        für mehrere unterschiedliche und z.\,T. voneinander unabhängige Konzepte wie Trennungsmuster, zu nutzende
        Fonts etc. steht}
      Bitte beachten Sie, dass ein mehrsprachiges Dokument aus der Sicht von
      \thepkg\ ein monosprachiges Dokument aus der Sicht von Babel sein kann.
      Allerdings können durchaus noch weitere Sprachen hinzukommen:
      Beispielsweise kann ein Dokument mit den Auswahlsprachen Deutsch und
      Englisch im Text lateinische Wörter oder Sätze enthalten.
      Diese Auswahl wird über die üblichen Babelmechanismen wie
      \cs*{selectlanguage}, \cs*{foreignlanguage} oder \cs*{text}\meta{sprache}
      vorgenommen.
      Wenn Babel oder Polyglossia geladen wird, was auf Wunsch auch direkt über
      \thepkg\ erfolgen kann, wird als aktive Sprache die Zielsprache ausgewählt.
    }{
       The language in which text fragments or individual words are actually written.
       This is the concept of language used by packages such as Babel or
       Polyglossia.\footnote{However, in these packages, "`language"' also refers to
        several different and, in some cases, independent concepts such as
        hyphenation patterns, fonts to be used, 
        etc.}
      Please note that a multilingual document from the perspective of
      \thepkg\ can be a monolingual document from the perspective of Babel.
      However, additional languages may well be added:
      For example, a document with the selected languages German and
      English may contain Latin words or sentences in the text.
      This selection is made using the usual Babel mechanisms such as
      \cs*{selectlanguage}, \cs*{foreignlanguage} or \cs*{text}\meta{language}.
      When Babel or Polyglossia is loaded, which can also be done directly via
      \thepkg\ if desired, the target language is selected as the active language.
    }
\end{enumerate}
\ldeen{
  Sprachen werden in \thepkg über ISO 639-1-Codes (z.\,B. "`en"', "`fr"',
  "`de"', \ldots) bezeichnet.
  Wenn für die Sprachbehandlung auf der Ebene von Babel/Polyglossia ein
  bestimmter Dialekt oder eine bestimmte Variante gewünscht ist, kann ein
  Mapping erfolgen.
}{
  Languages are designated in \thepkg using ISO 639-1 codes ("`english"', "`french"',
  "`ngerman”' \ldots).
  If a specific dialect or variant is desired for language processing at the Babel/Polyglossia level,
  mapping can be performed.
}
\section{\ldeen{Anwendung}{Usage}}
\label{sec:usage}
\ldeen{Das Paket wird auf die übliche Weise geladen}{The package is loaded in
  the usual way}:

\centerline{\cs*{usepackage}\oarg{options}\Marg{\packagename}}

\noindent
\ldeen*{%
  Da die Auswahl der Zielsprache sowohl über Optionen als auch auf anderem Weg
  erfolgen kann, werden hier zunächst diese verschiedenen Wege beschrieben.
  Weitere Optionen wie das Setzen der Auswahlsprachen werden in Abschnitt~@1 besprochen.%
}{%
  Since the target language can be selected both via options and in other ways,
  these different methods are described here first.
  Further options, such as setting the selectable languages, are discussed in
Section~@1.%
}{\ref{sec:options}}
\subsection{\ldeen{Auswahl der Zielsprache}{Selection of the Target Language}}
\label{sec:targetlang}

\ldeen{%
  Um eine Zielsprache für einen \LaTeX-Übersetzungslauf zu bestimmen, gibt es verschiedene
  Möglichkeiten. Die Reihenfolge in der folgenden Liste spiegelt auch die
  Reihenfolge der Auswertung. Sobald damit eine Zielsprache ermittelt wird, wird
  eine weitere Auswertung abgebrochen.
}{%
  There are various ways to determine a target language for a LaTeX translation run.
  The order in the following list also reflects the order of evaluation.
  Once a target language has been determined, further evaluation is canceled.
}
\begin{enumerate}
  \item \ldeen{Definition eines Makros \cs{omdTargetLanguage} \emph{vor} Laden des
    Pakets. Damit ist es beispielsweise möglich, über}{Definition of a macro
    \cs{omdTargetLanguage} \emph{before} loading the package. This makes it
    possible, for example, to use}  
\begin{verbatim}
> latexmk -e"\def\olsTargetLanguage{fr}" document.tex
\end{verbatim}
    \ldeen{die Zielsprache Französisch für den Übersetzungslauf festzulegen}{to
      set the target language French for the translation run}.
  \item \ldeen{Setzen der Paketoption \key{targetlang}{Zielsprache}}{Setting of
      the packet option \key{targetlang}{target language}}
  \item \ldeen{%
      Wenn die Option \keyis{targetlang}{job=\meta{n}} gesetzt wird, wird die
      Zielsprache aus dem $n.$ Element des Jobnamens bestimmt.
      Ein Element ist ein mit "`-"' (Minus) abgetrennter Teil des Jobnamens, siehe
      die Dokumentation zum Paket \needpackage{varsfromjobname}, das dafür geladen
      wird.
    }{%
      If the option \keyis{targetlang}{job=\meta{n}} is set, the
      target language is determined from the $n$th element of the job name.
      An element is a part of the job name separated by “`-”' (minus), see
      the documentation for the package \needpackage{varsfromjobname}, which is loaded
      for this purpose.%
    }%
    \ldeen{%
      Beispielsweise kann bei \keyis{targetlang}{job=2} über 
    }{%
      For example, with \keyis*{targetlang}{job=2} one can set Russian as target
      language using
    }
\begin{verbatim}
> latex -jobname doc-ru doc.tex
\end{verbatim}
    \ldeen{
      Russisch als Zielsprache eingestellt werden.
    }{}
  \item \ldeen{%
      Mit Setzen von \keyis*{targetlang}{meta} wird die in \cs*{DocumentMetaData}
      angegebene Sprache als Zielsprache verwendet.
    }{%
      Setting \keyis*{targetlang}{meta} uses the language specified in \cs*{DocumentMetaData}
      as the target language.
    }
  \item \ldeen{Wenn alles andere versagt, wird Englisch als Zielsprache
      gesetzt.}{If everything else fails, English is set as the target language.}
\end{enumerate}
\subsection{\ldeen{Optionen}{Options}}
\label{sec:options}
Es können folgende Optionen gesetzt werden:
\begin{options}
    \keyval{languages}{\ldeen{Liste von Auswahlsprachen}{list of selectable
        languages}}\Default!{}
    \ldeen*{Gibt zwei oder drei Auswahlsprachen an. Die Reihenfolge ist relevant für die Generierung
      der Sprachmakros (siehe Abschnitt~@1).
    }{%
      Provides two or three selectable languages.
      The order matters for the generation of the language commands
    (cf.\ Section~@1).
  }{\ref{sec:usermacros}}
  \keychoice{targetlang}{\meta{\ldeen{Sprache}{language}},\{job=\meta{n}\},meta}%\Default{}
  \ldeen{Siehe Abbschnitt}{See Section}~\ref{sec:targetlang}
  \keyval{prefix}{prefix}\Default{l}
  \ldeen{Setzt einen Prefix für die Multisprachmakros. Es kann auch ein leerer
    Prefix gesetzt werden.}{Sets a prefix for multilingual macros. An empty
    prefix can also be set.}
  \keybool{auto}\Default{true}\ldeen*{Die eigentlichen Sprachmakros werden
    angelegt, siehe Abschnitt~@1. Wird \keyis*-{auto}{false}
    gesetzt, werden nur die Makros zur Erzeugung der Sprachmakros zur Verfügung
    gestellt, nicht die Sprachmakros selbst.}{The actual language macros are
    created, see Section~@1. If \keyis*-{auto}{false}
    is set, only the macros for generating the language macros are provided, not
    the language macros themselves.}{\ref{sec:usermacros}}
\end{options}
   
\section{\ldeen{Nutzermakros}{User Macros}}
\label{sec:usermacros}
\ldeen{Wenn Sie nicht die Option \keyis-{auto}{false} gesetzt haben, wird ein
  Multisprachmakro angelegt. Dessen Name wird durch die Liste der Auswahlsprachen
  festgelegt. Je nachdem, ob Sie zwei oder mehr Auswahlsprachen angegeben haben,
  ist ist der Name}{If you have not set the option \keyis-{auto}{false}, a multilanguage macro will be created.
  Its names is determined by the list of selection languages.
  Depending on whether you have specified two or more selection languages,
  the name is}
\begin{itemize}[nosep]
  \item \cs{\meta{\ldeen{Prefix}{prefix}}\meta{\ldeen{Sprache}{language}$_{1}$}%
      \meta{\ldeen{Sprache}{language}$_{2}$}\marg{\ldeen{Text}{text}$_{1}$}\marg{\ldeen{Text}{text}$_{2}$}}
\end{itemize}
\ldeen{oder}{or}
\begin{itemize}[nosep]
  \item \cs{\meta{\ldeen{Prefix}{prefix}}\meta{\ldeen{Sprache}{language}$_{1}$}%
      \meta{\ldeen{Sprache}{language}$_{2}$}\meta{\ldeen{Sprache}{language}$_{3}$}%
      \marg{\ldeen{Text}{text}$_{1}$}\marg{\ldeen{Text}{text}$_{2}$}\marg{\ldeen{Text}{text}$_{3}$}
    }

\end{itemize}
\ldeen{Beispielsweise für für diese Dokumentation das Paket mit folgenden
  Optionen geladen:}{For example, for this documentation, the package was loaded with the following
  options:}
\begin{example}[code-only]
    \usepackage[
    languages={de,en},
    targetlang={job=2}
    ]{langselect}
\end{example}
\ldeen{Damit steht das Makro \cs*{ldeen\marg{Text$_{1}$}\marg{Text$_{1}$}} zur Verfügung.
  Wenn jetzt die Zielsprache "`de"' ausgewählt wird (hier über den Jobnamen,
  aber es gibt auch andere Möglichkeiten, vgl.\ Abschnitt~@1), wird nur
  \meta{Text$_{1}$} in das PDF-Dokument übernommen und \meta{Text$_{2}$}
  verworfen.
  Bei der Zielsprache "`en"' ist es umgekehrt.
}{This makes the macro \cs*{ldeen} available.

  If the target language "`de"' is now selected (here via the job name,
  but there are also other options, see Section~@1), only
  \meta{Text$_{1}$} is transferred to the PDF document and \meta{Text$_{2}$}
  is discarded.
  With the target language "`de"', it is the other way
  around.}{\ref{sec:targetlang}}
\begin{mdframed}[style=cnltxmdframed]
    \textbf{\ldeen{Achtung}{Attention}!}\par

    \noindent
    \ldeen*{Die automatische Generierung der Sprachmakros kann manchmal zu
      Problemen führen. Beispielsweise ergibt Abchasisch (Code: "`ab"' ) als Erst- und
      Griechisch (Code: "`el"') als Zweisprache den automatischen Namen \cs{label}, was zu
      Fehlermeldungen führt. Nutzen Sie in solchen Fällen die im Abschnitt~@1
      beschriebenen Möglichkeiten.}{The automatic generation of language macros can sometimes cause
      problems. For example, Abkhaz (code: "`ab"')  as the first language and
      Greek (Code: "`el"') as the second language results in the automatic name \cs{label}, which leads to
      error messages. In such cases, use the features described in section~@1.}{\ref{sec:own-names}}
\end{mdframed}
\subsection{\ldeen{Sternvariante}{Stared Variant}}
\label{sec:start-var}
\ldeen{Ein häufiges Problem bei mehreren Sprachversionen ist die Konsistenz
  nichtsprachabhängiger Daten, insbesondere wenn diese Daten sich öfter
  ändern. (Denken Sie \zB an Stundenpläne.)
  
  Für solche Fälle stellt \thepkg\ eine Sternvariante der Sprachmakros bereit.
  In dieser Sternvariante können die Textargumente Platzhalter der Form
  "`@\meta{n}"' enthalten, wobei \meta{n} einen Wert zwischen $1$ und $9$
  darstellt.
  Hinter dem Makro müssen dann weitere Argumente angegeben werden, und zwar so
  viele, wie es dem höchsten verwendeten \meta{n} entspricht. Die Sternvariante
  ersetzt die Platzhalter mit den entsprechenden Argumenten.
}{A common problem with multiple language versions is the consistency of
  non-language-dependent data, especially if this data changes frequently
  (think, for example, of timetables).
  
  For such cases, \thepkg\ provides a star variant of the language macros.
  In this star variant, the text arguments can contain placeholders of the form
  “`@\meta{n}”' can be used, where \meta{n} represents a value between $1$ and $9$.

  Additional arguments must then be specified after the macro, namely as many
  as correspond to the highest \meta{n} used. The star variant
  replaces the placeholders with the corresponding arguments.
}
\begin{example}[compile]
    \documentclass{article}
    \usepackage[languages={en,fr},
    targetlang=fr]{langselect}
    
    \begin{document}
    \lenfr*{The formula @1 generally refers to the Pythagorean theorem.}{La
      formule @1 fait généralement référence au théorème de
      Pythagore.}{$a^{2}+b^{2}=c^{2}$}
    \lenfr*{Calculate the length of the third side if one side is @1 units long
      and the hypotenuse is @2 units long.}{Calculez la longueur du troisième
      côté si un côté mesure @1 unités et l'hypoténuse @2 unités.}{$4$}{$5$}  
    \end{document}
\end{example}
\subsection{\ldeen{Individuelle Sprachmakros}{Individual Language Macros}}
\label{sec:own-names}
\ldeen{Wenn Ihnen die generierten Namen der Sprachmakros nicht gefallen oder
  diese sogar zu Kollisionen führen, können Sie diese ändern. Am einfachsten ist
  es, die Paketoption \option{prefix} zu ändern, so dass beispielsweise mit
  \keyis-{prefix}{mv} das Sprachmakro für die Deutsch/Englisch-Kombination
  \cs{mvdeen} heisst.}{If you do not like the generated names of the language
  macros or if they even lead to collisions, you can change them. The easiest
  way is to change the package option \option{prefix} so that, for example, with
  \keyis-{prefix}{mv}, the language macro for the German/English combination
  is called \cs{mvdeen}.}
\section{\ldeen{Beispiele}{Examples}}
\section{Implementation}
\AutoDocInput{\packagename.dtx}
\end{document}
%</driver>

%<*pkg>
\NeedsTeXFormat{LaTeX2e}[2022/06/01]
\def\packagename{langselect}
\def\packageversion{2025/08/10 v0.9.0d}
\ProvidesPackage{\packagename}[\packageversion\space support for different
languages versions of a document]
\ExplSyntaxOn
% \ldeen{\cs{olsIsoTags} enthält alle gültigen ISO
% 639-1-Codestags.}{\cs{olsIsoTags} contains all valid ISO 639-1 code tags.}
\seq_const_from_clist:cn{olsIsoTags}{aa,ab,ae,af,ak,am,an,ar,as,av,ay,az,ba,be,bg,
  bh,bi,bm,bn,bo,br,bs,ca,ce,ch,co,cr,cs,cu,cv,cy,da,de,dv,dz,ee,el,en,eo,es,et,eu,
  fa,ff,fi,fj,fo,fr,fy,ga,gd,gl,gn,gu,gv,ha,he,hi,ho,hr,ht,hu,hy,hz,ia,id,ie,ig,ii,
  ik,io,is,it,iu,ja,jv,ka,kg,ki,kj,kk,kl,km,kn,ko,kr,ks,ku,kv,kw,ky,la,lb,lg,li,ln,
  lo,lt,lu,lv,mg,mh,mi,mk,ml,mn,mr,ms,mt,my,na,nb,nd,ne,ng,nl,nn,no,nr,nv,ny,oc,oj,
  om,or,os,pa,pi,pl,ps,pt,qu,rm,rn,ro,ru,rw,sa,sc,sd,se,sg,si,sk,sl,sm,sn,so,sq,sr,
  ss,st,su,sv,sw,ta,te,tg,th,ti,tk,tl,tn,to,tr,ts,tt,tw,ty,ug,uk,ur,uz,ve,vi,vo,wa,
  wo,xh,yi,yo,za,zh,zu} 

\newcommand\IfTagIsValidF[2]{%
  \seq_if_in:ceF{olsIsoTags}{#1}{#2}
}
\RequirePackage{varsfromjobname}[2025/08/03]
% \ldeen{Das Ergebnisse von \cs{varsfromjobname}\ und \cs{str\_range:Nnn} haben
% Catcode 12, aber Iso-Tags und Optionen haben Catcode 11. Wir vereinheitlichen
% durch einen Wrapper auf Catcode 11.}{The results of \cs{varsfromjobname}\
% and \cs{str\_range:Nnn} have catcode 12, but ISO tags and options have catcode
% 11. We standardize using a wrapper on catcode 11.} 
\newcommand\olsChangeCatEleven[1]{
  \def\tmpa{#1} 
  \edef\tmpb{
    \noexpand\scantokens{
      \noexpand\edef\noexpand\olsCatElven{\tmpa}
    }
  }
  \tmpb
}
% \ldeen{Es gibt nur eine Handvoll Optionen.}{There are only a handful of options.}
\DeclareKeys{
  languages.clist_gset:N = \ols_langs,
  languages.usage=load,
  targetlang.code = {
    \cs_if_exist:NF\olsTargetLanguage{
      \str_if_eq:eeTF{\str_range:Nnn{#1}{1}{3}}{job}{% 
        \olsChangeCatEleven{\getfromjobname{\str_range:Nnn{#1}{-1}{-1}}}
        \edef\olsTargetLanguage{\olsCatElven}
      }{
        \str_if_eq:nnTF{#1}{meta}{
          \IfDocumentMetadataTF{
            \edef\tmpa{\GetDocumentProperties{document/lang}}
            \edef\tmpb{\str_range:Nnn\tmpa{1}{2}}
            \olsChangeCatEleven{\tmpb}
            \edef\olsTargetLanguage{\olsCatElven}
          }{
            \PackageWarningNoLine{\packagename}{No~meta~data~provided,
              ~falling~back~to~English}
            \def\olsTargetLanguage{en}
          }
        }{
          \edef\olsTargetLanguage{#1}
        }
      }
    }{}
  },
  targetlang.usage=load,
  prefix.store=\ols_sprefix,
  prefix.initial:n=l,
  prefix.usage=load,
  auto.bool_set:N=\ols_generate,
  auto.usage=load,
  auto.initial:n=true
}%
\ProcessKeyOptions\relax
\cs_if_exist:NF \olsTargetLanguage {\def\olsTargetLanguage{en}}
% \ldeen{
% Wir führen verschiedene Checks durch. Als erstes prüfen wir, ob eine valide
% Zielsprache ermittelt werden konnte.
% }{
% We are doing some checks. First, we check whether a valid
% target language could be determined.
% }
\IfTagIsValidF{\olsTargetLanguage}{
  \PackageWarningNoLine{\packagename}{Can't~resolve~target~language~'\olsTargetLanguage'.
    \MessageBreak Fall~back~to~'en'}
  \def\olsTargetLanguage{en}
}
% \ldeen{Sind mindestes zwei Auswahlsprachen angegeben?}{Are at least two
% selectable languages specified?}
\int_compare:nNnT{\clist_count:N{\ols_langs}} < {2}{
  \PackageError{\packagename}{No~sufficient~number~of~selectable
    \MessageBreak~languages~provided}{
    You~have~to~provide~a~list~of~two~or~three\MessageBreak
    languages~via~'language'~option.}
  \aftergroup\endinput      
}
% \ldeen{%
% Im Moment kann \thepkg\ nicht mehr als drei Auswahlsprachen verarbeiten. Wir
% warnen daher, wenn mehr als drei Sprachen angegeben werden.}{%
% Currently, \thepkg\ can't process more than three languages. Thus, we issue a
% warning if the list of selectable languages contrains more than three languages.
%}
\int_compare:nNnT{\clist_count:N\ols_langs} > {3}{
  \PackageWarningNoLine{\packagename}{Too~many~selectable~languages~provided.
    \MessageBreak I~will~ignore~the~superflous~languages}
}
% \ldeen{Als nächstes überprüfen wir, ob alle Auswahlsprachen gülitge ISO
% 639~1 Codes sind.}{Next, we test the validity of the selectable
% languages. They should form ISO~693-1 codes.} 
\clist_map_inline:Nn\ols_langs {
  %\typeout{***~CHECK~'#1'}
  \IfTagIsValidF{#1}{
    \PackageError{\packagename}{Couldn't~resolve~selectable~language~'#1'}{
      Use~valid~ISO-639-1~code~in~option~'languages'.}
    \aftergroup\endinput
  }
}
% \ldeen{Die Zielsprache sollte in der Liste von Auswahlsprachen enthalten
% sein.}{The list of selectable languages should include the target language.}
\clist_if_in:NVF\ols_langs{\olsTargetLanguage}{
  \PackageError{\packagename}{Target~language~'\olsTargetLanguage'~is~not\MessageBreak
    in~the~list~of~selectable~languages}{
    Check~option~'language'~and~'targetlang'.}
}
% \ldeen{Wir korrigieren noch ggf.\ die über \cs{DocumentMetadata} gesetzen
% Metadaten auf die der Zielsprache, so dass beispielsweise Babel
% keine falsche Hauptsprache ableitet.
% }{%
% If needed, we correct the language metadata set by \cs{DocumentMetadata}.
% Now, Babel (for example) does not derive a wrong main language.
%}
\IfDocumentMetadataT{
  \edef\olsDocLang{\exp_args:Ne\str_range:nnn{\GetDocumentProperties{document/lang}}{1}{2}}
  \str_if_eq:eeF{\olsTargetLanguage}{\olsDocLang}{
    \PackageWarningNoLine{\packagename}{Target~language~'\olsTargetLanguage'~
      doesn't~comply~with\MessageBreak \c_backslash_str
      DocumentMetadata~'\GetDocumentProperties{document/lang}'.\MessageBreak
      I'll~try~to~overwrite~the~meta~data}
    \AddToDocumentProperties[document]{lang}{\olsTargetLanguage}
  }
}
% \ldeen{Ein Boolean zur Markierung, ob mehr als zwei Auswahlsprachen angegeben
% wurden.}{A Boolean to indicate whether more than two selectable languages have
% been specified.} 
\bool_new:c{ols_trilang}
\int_compare:nNnTF{\clist_count:N{\ols_langs}} > {2}{
  \bool_set_true:c{ols_trilang}
}{
  \bool_set_false:c{ols_trilang}
}
\def\olsFrstLanguage{\clist_item:Nn \ols_langs{1}}
\def\olsScndLanguage{\clist_item:Nn \ols_langs{2}}
\bool_if:cT{ols_trilang}{
  \def\olsThrdLanguage{\clist_item:Nn \ols_langs{3}}
}
% \ldeen{Die Makros \cs{olsMakeBilangualMacro} bzw.\ \cs{olsMakeTrilangualMacro}
% dienen zur Generierung der eigentlichen Sprachmakros.}{The commands
% \cs{olsMakeBilangualMacro} and \cs{olsMakeTrilangualMacro}, resp., 
% are used to generate the actual language macros.}
 % #1: prefix, #2,#3: selectable languages, #4: own name
\NewDocumentCommand{\olsMakeBilangualMacro}{O{\ols_sprefix} m m o}{
  \str_if_eq:eeT{\olsTargetLanguage}{#2} {
    \IfValueTF{#4}{
      \ExpandArgs{c}\NewDocumentCommand{#1#4}{s +m +m}{
        \IfBooleanTF{##1}{
          \olsProcessArgs{##2}
        }{
          ##2
        }
      }
    }{
      \ExpandArgs{c}\NewDocumentCommand{#1#2#3}{s +m +m}{
        \IfBooleanTF{##1}{
          \olsProcessArgs{##2}
        }{
          ##2
        }
      }
    }
  }
  \str_if_eq:eeT{\olsTargetLanguage}{#3}{
    \IfValueTF{#4}{
      \ExpandArgs{c}\NewDocumentCommand{#1#4}{s +m +m}{
        \IfBooleanTF{##1}{
          \olsProcessArgs{##3}
        }{
          ##3
        }
      }
    }{
      \ExpandArgs{c}\NewDocumentCommand{#1#2#3}{s +m +m}{
        \IfBooleanTF{##1}{
          \olsProcessArgs{##3}
        }{
          ##3
        }
      }
    }
  }
}
% \ldeen{Nun das Gleiche nochmal für \emph{drei} Sprachen.}{Now do the same thing again for three languages.}
\NewDocumentCommand{\olsMakeTrilangualMacro}{O{\ols_sprefix} m m m o}{
  \str_if_eq:eeT{\olsTargetLanguage}{#2} {
    \IfValueTF{#4}{
      \ExpandArgs{c}\NewDocumentCommand{#1#5}{s +m +m +m}{
        \IfBooleanTF{##1}{
          \olsProcessArgs{##2}
        }{
          ##2
        }
      }
    }{
      \ExpandArgs{c}\NewDocumentCommand{#1#2#3}{s +m +m +m}{
        \IfBooleanTF{##1}{
          \olsProcessArgs{##2}
        }{
          ##2
        }
      }
    }
  }
  \str_if_eq:eeT{\olsTargetLanguage}{#3} {
    \IfValueTF{#4}{
      \ExpandArgs{c}\NewDocumentCommand{#1#5}{s +m +m +m}{
        \IfBooleanTF{##1}{
          \olsProcessArgs{##2}
        }{
          ##3
        }
      }
    }{
      \ExpandArgs{c}\NewDocumentCommand{#1#2#3}{s +m +m +m}{
        \IfBooleanTF{##1}{
          \olsProcessArgs{##2}
        }{
          ##3
        }
      }
    }
  }
  \str_if_eq:eeT{\olsTargetLanguage}{#4} {
    \IfValueTF{#4}{
      \ExpandArgs{c}\NewDocumentCommand{#1#5}{s +m +m +m}{
        \IfBooleanTF{##1}{
          \olsProcessArgs{##2}
        }{
          ##4
        }
      }
    }{
      \ExpandArgs{c}\NewDocumentCommand{#1#2#3}{s +m +m +m}{
        \IfBooleanTF{##1}{
          \olsProcessArgs{##2}
        }{
          ##4
        }
      }
    }
  }
}
% \ldeen{Falls die automatische Erzeugung nicht deaktivert ist, werden die
% konkreten Sprachmakros angelegt. Der Name sind
% \cs{\meta{Prefix}\meta{Sprache$_{1}}$\meta{Sprache$_{2}$}}
% bzw. \cs{\meta{Prefix}\meta{Sprache$_{1}}$\meta{Sprache$_{2}$}\meta{Sprache$_{2}$}}.}{%
% If automatic generation is not disabled, the
% specific language macros are created. The names are
% \cs{\meta{prefix}\meta{language$_{1}$}\meta{Language$_{2}$}}
% or \cs{\meta{prefix}\meta{language$_{1}$}\meta{language$_{2}$}\meta{language$_{3}$}}.
% }
\bool_if:cTF{ols_generate}{
  \bool_if:cTF{ols_trilang}{
    \olsMakeTrilangualMacro{\olsFrstLanguage}{\olsScndLanguage}{\olsThrdLanguage}
  }{
    \olsMakeBilangualMacro{\olsFrstLanguage}{\olsScndLanguage}
  }
}{
  \PackageInfoNoLine{\packagename}{No~language~macros~generated.\MessageBreak
  Use~\string\olsMakeBilangualMacro~or~\string\olsMakeTrilangualMacro~to~generate.}
}
% \ldeen{Für die Sternvarianten der Sprachmakros legen wir uns ein Makro an,
% dass die folgenden Argumente verarbeitet.}{For the star variants of the
% language commands, we create a command that processes the subsequent
% arguments.} 
\seq_new:N \ols_args_seq 
\int_new:N \ols_num_of_args
\tl_new:N \ols_format_str_tl

% \ldeen{Da das Platzhalterzeichen @ im folgendem Code als Literal auftritt,
% muss es den gleichen Catcode wie im Dokument haben.}{Since the placeholder
% character @ appears as a literal in the following code, 
% it must have the same catcode as in the document.}
\makeatother
\NewDocumentCommand \olsProcessArgs { +m } {
  \tl_gset:Nn \ols_format_str_tl{#1}
  \int_zero:N \ols_num_of_args
% \ldeen*{Suche den höchsten Wert @1, für den @@1 im Formatstring vorkommt.}{Find
% the highest value @1, where @@1 is contained in the format string.}{$n$}  
  \int_step_inline:nnnn { 9 } { -1 } { 1 }
  {
    \regex_if_match:nnT {@##1}{ #1 }
    {
      \int_set:Nn \ols_num_of_args { ##1 } \prg_break:
    }
  }
% \ldeen{Wenn kein Platzhalterzeichen gefunden wurde, gib den Text in
% Originalform aus. Ansonsten wird das Ersetzungmakro gerufen.}{If no
% placeholder character is found, output the text in 
% original form. Otherwise, the replacement macro is called.}  
  \int_compare:nNnTF {\ols_num_of_args} = { 0 }
  { #1 }{
    \seq_clear:N \ols_args_seq
    \ols_collect_replace:n \ols_num_of_args
  }
}
% \ldeen*{@1 erhält \emph{ein} Argument übergeben, nämlich die Anzahl der
% verbleibenden Ersetzungstexte. Es konsumiert aber \emph{zwei}
% Argumente. Dieses zweite holt es sich aus dem nach dem Makro liegenden
% Tokenstrom, wo es den jeweilig nächsten Ersetzungstext findet.
% Dieser wird in einer Tokensequent gespeichert und @1 erneut aufgerufen.}{@1
% receives \emph{one} argument, namely the number of remaining replacement
% texts. However, it consumes \emph{two} arguments. It obtains the second one
% from the token stream following the macro, where it finds the next replacement
% text. This is stored in a token sequence and @1 is called again.}{\cs{ols\_collect\_replace}}
\cs_new_protected:Npn \ols_collect_replace:n #1 #2{
  \seq_put_right:Nn \ols_args_seq { #2 }
  \int_compare:nNnTF { #1 } = { 1 }
  {
% \ldeen{Wenn alle Ersetzungstexte eingelesen wurden, werden sie nacheinander
% auf die Platzhalterausdrücke angewendet.}{Once all replacement texts have been
% read in, they are applied one after the other 
% to the placeholder expressions.}
    \int_step_inline:nn { \ols_num_of_args }
    {
      \tl_replace_all:Nnn \ols_format_str_tl
      { @##1 }
      { \seq_item:Nn \ols_args_seq { ##1 } }
    }
    \ols_format_str_tl
  }{
    \ols_collect_replace:n { \int_eval:n { #1 - 1 } }
  }
}
\makeatletter
\ExplSyntaxOff

%</pkg>
% ----------------- Ende Paketcode ----------------------------
% \section{Anwendung}
% Laden und verwenden:
%
% \begin{verbatim}
% \usepackage{mypkg}
% \end{verbatim}
%
% Das war's.
